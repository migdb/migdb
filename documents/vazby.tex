\documentclass[11pt,a4paper]{article}   	
\usepackage[utf8]{inputenc}              	
\usepackage[czech]{babel}                	
\usepackage[pdftex]{graphicx}
\usepackage{amsfonts}				
\begin{document}
\section{Mapování vazeb}
	\subsection{Úvod - upřesnění formulace}
	Zápisem tohoto typu:\\ 
	\texttt	{SELF.b.isOpposite.isCollection.isUnique \&\& SELF.b.a.isCollection.isUnique; \\} 
	Mám na mysli následující:\\
	\texttt{b} má protějšek a zároveň je unykátní kolekcí a zároveň protějšek \texttt {a} je unikátní kolekcí.
	Ze zápisu tedy vyplývá, že vypisuji pouze TRUE vlastnosti danný property. 		  							
	                                
	\subsection{Vlastnosti mapované na prázdnou množinu}
   		Jedná se o dvě základní vlastnosti tříd. Pokud je třída Embedded, nebo
   		Transient, namapují se jejich atributy na prázdnou množinu. 
   		\begin{itemize}				    
         	\item \texttt	{
         					$\pi_0(b) \to \{ 0\}$  
         					}
   		\end{itemize}
   	\subsection{Jednostranně navigabilní vazby}
   		Třída A vidí zkrz svůji property na trídu B, ale B nemá žádnou takovou property a 
   		tak třídu A nevidí a ani neví, že je třídou A viděn. Při jednostranně navigabilním
   		mapování sice třída B třídu A nevidí, ale v databázi na ni má uloženy FK.
   		\\
   		Máme dva případy mapování:
   		\begin{itemize}				    
         	\item \texttt	{SELF.b.isCollection.isOrdered; \\}
         					(Toto mapování obsahuje i sloupec s pořadím prvků) \\
         					$\pi_1(b) \to \{ sloupec\}$
         					
         	\item \texttt	{SELF.b.isCollection.isUnique; \\
         					 SELF.b.isCollection; \\
         					$\pi_2(b) \to \{ sloupec\}$
         					}
         	\item Jako třetí případ můžeme považovat \texttt   {SELF.b}. Tento případ je ale stejný, 
         	jako u případů oboustraně navigabilních, tak bude mezi nimi.
         		  							
   		\end{itemize}
   	\subsection{Oboustranně navigabilní vazby}
   		Třída A vidí zkrz svůji property na trídu B a třída vidí skrz svou property na třídu A. \\
   		Máme sedm případů mapování, které je asi nejlepší rozdělit podle toho, zda je \texttt   {SELF.b.isOpposite} kolekce, či ne.
   		\subsubsection{SELF.b.isOpposite - isCollection = FALSE}
   		V tomto případě máme obecně už jen dva možné výstupy:
   		   		\begin{itemize}				    
         			\item \texttt	{SELF.b.a; \\
         							 SELF.b; \\
         							 SELF.b.a.isCollection;\\
         							 SELF.b.a.isCollection.isUnique;\\
         							$\pi_3(b) \to \{ sloupec, FK\}$ \\
         							$\pi_3(a) \to \{ 0\}$
         							}
         			\item \texttt	{SELF.b.a.isCollection.isOrdered; \\}
         							(Toto mapování obsahuje v \texttt {b} i sloupec s pořadím prvků) \\
         							$\pi_4(b) \to \{ sloupec, FK\}$ \\
         							$\pi_4(a) \to \{ 0\}$  		  							
   				\end{itemize}
   		\subsubsection{SELF.b.isOpposite - isCollection = TRUE}
   		Tato část je poměrně obsáhlá, ale ve finále vede pouze na pět různých mapování:
   				\begin{itemize}				    
         			\item \texttt	{SELF.b.isOpposite.isCollection.isOrdered \&\& SELF.b.a; \\}
         							(Toto mapování obsahuje v \texttt {b} i sloupec s pořadím prvků) \\        							
         							$\pi_5(b) \to \{ sloupec\}$ \\
         							$\pi_5(a) \to \{ sloupec, FK\}$		
         			\item \texttt	{SELF.b.isOpposite.isCollection.isOrdered \&\& SELF.b.a.isCollection.isOrdered; \\}
         							(Toto mapování obsahuje v \texttt {b, a} i sloupec s pořadím prvků) \\        							
         							$\pi_6(b) \to \{ sloupec\}$ \\
         							$\pi_6(a) \to \{ sloupec\}$	\\	
         							(Vznik vazební tabulky s cizími klíči)
         			\item \texttt	{SELF.b.isOpposite.isCollection.isUnique \&\& SELF.b.a; \\
         							 SELF.b.isOpposite.isCollection \&\& SELF.b.a; \\}    							
         							$\pi_7(b) \to \{ 0\}$ \\
         							$\pi_7(a) \to \{ sloupec, FK\}$	\\
         			\item \texttt	{SELF.b.isOpposite.isCollection.isUnique \&\& SELF.b.a.isCollection.isOrdered; \\
         							 SELF.b.isOpposite.isCollection \&\& SELF.b.a.isCollection.isOrdered; \\
         							 SELF.b.isOpposite.isCollection.isOrdered \&\& SELF.b.a.isCollection.isUnique; \\
         							 SELF.b.isOpposite.isCollection.isOrdered \&\& SELF.b.a.isCollection; \\}
         							(Toto mapování obsahuje v \texttt {a} i sloupec s pořadím prvků) \\        							
         							$\pi_8(b) \to \{ 0\}$ \\
         							$\pi_8(a) \to \{ sloupec\}$	\\	
         							(Vznik vazební tabulky s cizími klíči)	 
         		     \item \texttt	{SELF.b.isOpposite.isCollection.isUnique \&\& SELF.b.a.isCollection.isUnique; \\
         							 SELF.b.isOpposite.isCollection.isUnique \&\& SELF.b.a.isCollection; \\
         							 SELF.b.isOpposite.isCollection \&\& SELF.b.a.isCollection.isUnique; \\
         							 SELF.b.isOpposite.isCollection \&\& SELF.b.a.isCollection; \\}       							
         							$\pi_9(b) \to \{ 0\}$ \\
         							$\pi_9(a) \to \{ 0\}$	\\	
         							(Vznik vazební tabulky s cizími klíči)	 				   							
   				\end{itemize}			   		    
\end{document} %konec dokumentu